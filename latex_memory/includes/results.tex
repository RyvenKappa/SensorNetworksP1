\section{RESULTS}

A \acrshort{lorawan} based system was implemented from the case use study to the implementation, following all the specific requirements told by the teachers of the subject. 
To further expand the capabilities of the communications, a high amount of the efforts allocated were applied to design an advanced header and frame structure to utilize all the 
\texttt{30 Bytes} limits of \acrshort{lorawan}.

To achieve this in the time limit, a GitHub based continuous design and integration was followed. In relation to this, the project was documented with Doxygen , and deployed as a static page with GitHub actions, this information 
is available at \url{https://ryvenkappa.github.io/SensorNetworksP1/}, with the final repository of the project being at: \url{https://github.com/RyvenKappa/SensorNetworksP1}.

Finally, the project allowed the student to understand the limitations and capabilities of \acrshort{lorawan}, understanding as well the basis of the standard model design 
for \acrshort{iot} systems that integrate gateways for intercommunications.



