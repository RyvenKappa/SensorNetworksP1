\section{REQUIREMENTS ANALYSIS}
\subsection{Specifications required and implemented}

This section presents in the next tables the list of requirements needed to implement, as well as the implementation status for each specification.
\vspace{2\baselineskip}
\begin{table}[H]
    \begin{center}
        \begin{tabular}{|p{0.1\textwidth} | p{0.7\textwidth} | p{0.15\textwidth}|}
            \hline
            \textbf{Req. ID} & \textbf{Requirement description} & \textbf{Implemented}\\
            \hline
            Phase2.1 & The target must be able to send the location. & Yes\\
            \hline
            Phase2.2 & The target must be able to send, apart from the location, 3 environmental parameters. & Yes\\
            \hline
            Phase2.3 & The dashboard must have a widget to represent the previous parameters values. & Yes\\
            \hline
            Phase2.4 & The dashboard must allow the user to send commands to change the RGB Led. The commands should be: ``OFF'', ``Green'' and ``Red''.  & Yes\\
            \hline
        \end{tabular} 
    \end{center}
    \caption{Phase 2 requirements implementation status}
    \label{ReqGeneral}
\end{table}

\vspace{2\baselineskip}

\begin{table}[H]
    \begin{center}
        \begin{tabular}{|p{0.1\textwidth} | p{0.7\textwidth} | p{0.15\textwidth}|}
            \hline
            \textbf{Req. ID} & \textbf{Requirement description} & \textbf{Implemented}\\
            \hline
            Phase3.1 & Get all the parameters from the sensors. & Yes\\
            \hline
            Phase3.2 & Convert the data format of the system parameters from string to the most appropriate type to reduce their length. & Yes\\
            \hline
            Phase3.3 & Modify the LUA code to retrieve the values of all the parameters. & Yes\\
            \hline
        \end{tabular} 
    \end{center}
    \caption{Optional requirements implementation status}
    \label{ReqTest}
\end{table}

\subsection{Extra specifications implemented}

To further expand the communication capabilities of the system, a frame header was implemented. This allows for:
\begin{itemize}
    \item Frame version.
    \item Control and data separation.
    \item Scalability.
\end{itemize}

The design for this is described in \hyperref[advanced]{a dedicated chapter} of the document.